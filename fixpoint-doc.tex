\documentclass{article}

\setlength{\parskip}{\baselineskip}
\setlength{\parindent}{0 pt}

\usepackage{fullpage, xspace, enumitem}

\usepackage[pdfborder={.25 .25 .25}]{hyperref}

\urlstyle{rm}

\usepackage{microtype}

\newcommand{\blob}{\textbf{Blob}\xspace}
\newcommand{\blobs}{\textbf{Blob}s\xspace}
\newcommand{\valuex}{\textbf{Value}\xspace}
\newcommand{\valuexs}{\textbf{Value}s\xspace}
\newcommand{\encode}{\textbf{Encode}\xspace}
\newcommand{\thunk}{\textbf{Thunk}\xspace}
\newcommand{\thunks}{\textbf{Thunk}s\xspace}
\newcommand{\encodes}{\textbf{Encode}s\xspace}
\newcommand{\name}{\textbf{Name}\xspace}
\newcommand{\names}{\textbf{Name}s\xspace}

\begin{document}

\textbf{Computation in the Fixpoint OS}\newline
Yuhan Deng, Sadjad Fouladi, Keith Winstein

The intention behind Fixpoint was to follow the design of the SICP
\href{https://mitpress.mit.edu/sites/default/files/sicp/full-text/book/book-Z-H-27.html#\%_sec_4.2.2}{``Interpreter
  with Lazy Evaluation''} and support arbitrary computations on
content-addressed data, including workloads like software compilation,
unit testing, video encoding, one-off Python scripts that read a text
file and write another text file, big data-analysis jobs, massively
parallel jobs that benefit from 10,000 nodes, etc. We also want to
support small functions with
$<$50~ns overhead per invocation---meaning, without a context
switch. We chose Wasm as the basic representation of computation
because (1) LLVM can compile into it, which allows a huge amount of
existing software to be expressed, (2) it can run at near-native speed
(via \texttt{wasm2c} and a C compiler, reserving 8 GiB of virtual
address space per invocation to make it impossible for the function to
reference memory that doesn't belong to it), and (3) we can guarantee
that computations are deterministic functions of their inputs and can
memoize execution and apply transformations without altering the
results.

\vspace{0.5\baselineskip}
{\Large \textbf{1. Types}}

The basic type is a \valuex, an immutable vector of bytes. Each \valuex is one of two types:
\begin{enumerate}[topsep=0pt]
\item A \blob: an immutable vector of bytes with arbitrary contents

\item A \thunk: a reference to a \blob that will be the output of a computation
\end{enumerate}

A \name is a string that identifies a \valuex. A \name contains either:
\begin{enumerate}[topsep=0pt]
\item A ``canonical name'': the SHA-256 hash of the contents of a \valuex

\item A ``literal'': a \blob contained in the \name itself, or
  
\item A \thunk
\end{enumerate}

An \encode\footnote{For ``Explicit Named Computation On Data or
  Encodes.''} is a \blob that describes the application of a function
to a captured environment. It specifies:
\begin{enumerate}[topsep=0pt]

\item The ``function'', which is either:
\begin{itemize}[topsep=0pt]
\item the \name of a \blob that will be interpreted as a Wasm module in the binary format, along with the index of a function defined by the Wasm module, or
\item the \name of a \blob that will be interpreted as an \encode, with its inputs prepended to those in this \encode
\end{itemize}

\item The ``strict inputs'': a list of \names that designate the
  \blobs available to the function at runtime.\footnote{Because a
    \name can be a \thunk, the input can refer to a \blob that hasn't been computed yet.}
  
\item The ``lazy inputs'': a list of \names that designate \valuexs
  (not necessarily \blobs) that can be referenced in the function's
  outputs. Neither the \names nor the contents of the referenced
  \valuexs are accessible to the function, but the input can be
  referred to in the function's outputs.\footnote{This can support
    functions that do short-circuit evaluation, e.g. ``\texttt{if}''
    or ``\texttt{and},'' while retaining the system's ability to
    replace any \thunk with its \blob value at any point in a
    computation without affecting the results. That is, without
    full-blown fexprs.}

\end{enumerate}
  
The inputs are identified by an \texttt{index}: a non-negative integer
counting up from zero (the first input to the innermost
function). Strict inputs, lazy inputs, and outputs are numbered
separately.

\newpage
The format of a \thunk is:
\begin{enumerate}[topsep=0pt]
\item The \name of an \encode
\item A type indicating whether the reference is to an output, strict input, or lazy input of the \encode
\item A non-negative integer \texttt{index}
  \end{enumerate}

\vspace{0.5\baselineskip}
{\Large \textbf{2. State}}

The system maintains two mappings:
\begin{itemize}[topsep=0pt]
\item The ``storage'' maps a \name to the contents of the corresponding \valuex.

\item The ``memoization cache'' maps a \thunk to a set of \names (ideally including the canonical name of the resulting \blob).
\end{itemize}
  
\vspace{0.5\baselineskip}
{\Large \textbf{3. Operations}}

A \valuex can be ``forced.'' This is similar to what SICP calls the
``\texttt{force-it}'' procedure. Forcing a \blob is the identity
operation and does nothing. Forcing a \thunk retrieves its value: the
corresponding \blob.

To force a \thunk, the system can:
\begin{itemize}[topsep=0pt]
\item retrieve a memoized result from the memoization cache if available, or
\item compute it, by:
  \begin{enumerate}[topsep=0pt]
  \item evaluating the \encode that the \thunk refers to
  \item retrieving the identified output
  \item optionally inserting an entry into the memoization cache, and
  \item forcing the output in question until a \blob is the result
  \end{enumerate}
\end{itemize}

To evaluate an \encode, the system will:
\begin{enumerate}[topsep=0pt]
\item force all the strict inputs of the \encode (optionally in parallel), transforming each input into a \blob
\item apply the function to its inputs. This means running the Wasm function and allowing it to:
\begin{itemize}[topsep=0pt]
\item read the contents of any strict input
\item modify the contents of any strict input. The function doesn't
  know whether the runtime is implementing this by making a copy of
  the input, or by allowing the function to mutate its only copy
  (based on a guess that the \blob won't be referenced again
  and can be recomputed if needed).
\item insert arbitrary \valuexs into the storage, which are known as
  the \encode's ``outputs.'' To avoid computing the hash of every
  intermediate \valuex, the \valuex is inserted with a key that
  is a \thunk containing the \name of the \encode and an output
  \texttt{index} chosen by the function.
\item insert a \thunk into the storage that refers to one of the
  \encode's lazy inputs, even though the function doesn't know the \name of
  the \encode, or the \name or contents of the lazy inputs---this is
  necessary to preserve referential transparency and forbid cyclic references
\end{itemize}
\end{enumerate}

The storage is never \emph{required} to compute the SHA-256 hash of a
\valuex. However, to avoid storing duplicate long-lived \valuexs under
different keys, the runtime will probably choose to hash any
long-lived \valuex that is identified by a \thunk key, and replace the
key with a canonical name, moving the original \thunk name to an entry
in the memoization cache.

\vspace{0.5\baselineskip}
{\Large \textbf{4. Examples}}

\textbf{Example I: Simple addition.} To add two numbers (e.g., 4 and
7), the user can create an \texttt{add} function in Wasm that expects
two inputs (\#0 and \#1) and writes to an output with an
\texttt{index} of \#0. (We sometimes refer to output \#0 as the
value ``returned'' by the \encode, but a \thunk can reference any
output.) The user inserts this function into the storage under its
canonical name (the SHA-256 hash of the Wasm module).

The user then writes an \encode where:
\begin{enumerate}[topsep=0pt]
\item The function is the canonical name of the Wasm module (with the appropriate function index)
\item The two strict inputs are the literal ``4'' and the literal ``7''
\end{enumerate}
The user inserts this \encode into the storage under its canonical
name (the SHA-256 hash of the \encode).

The user then writes a \thunk referring to the just-inserted \encode and output index \#0, and asks Fixpoint
to force it. The result will be a \blob with the contents ``11''.

\textbf{Example II: Input dependencies.} To compile and link a C program with two source files,
the build system can create three \encodes:
\begin{enumerate}[topsep=0pt]
\item An \encode that preprocesses, compiles, and assembles the first source file to a \texttt{.o} file
  (the \texttt{.c} file and all system headers are strict inputs, identified as \blobs)
\item An \encode that does the same to produce the second \texttt{.o} file
\item An \encode that links the \texttt{.o} files and produces an executable. The strict inputs will include \thunks corresponding to the output of the first two \encodes, plus all necessary libraries.
\end{enumerate}

When the user forces a \thunk that refers to the output of the last \encode, the system will need to force
all its strict inputs before applying the function. This can compile each source file in parallel before linking.

\textbf{Example III: Tail recursion.} To compute the $n$th Fibonacci number
recursively, the user can write a \texttt{fib} function that
takes one numerical argument.
\begin{itemize}[topsep=0pt]
\item \texttt{fib(1)} returns ``1''
\item \texttt{fib(2)} returns ``1''
  \item \texttt{fib(n)} (for $n>2$) creates three \encodes
    representing \texttt{fib(n-1)}, \texttt{fib(n-2)}, and
    \texttt{add} applied to the outputs of the previous two, and
    returns a \thunk referring to the output of the \texttt{add}
    \encode.
    \end{itemize}

If the user forces a \thunk referring to the output of \texttt{fib(200)}, the system will
evaluate the \encode and the output will be another \thunk---which it will then force, in turn,
until finally reaching a \blob answer.

\textbf{Example IV: map operator.} The \texttt{map} function expects a
lazy argument naming a function $f$, and a lazy argument for each item
being mapped over.  It returns a \thunk representing the output of an
\encode that concatenates a series of strict inputs, each one a \thunk
representing the application of $f$ to one of the items.

\textbf{Example V: special forms.} To implement an \texttt{if}
operator, the user writes a function that expects one strict
input (the predicate) and two lazy inputs (the consequent and
alternative), and writes one output.

The function examines the contents of the ``predicate'' input,
interpreting it with whatever representation the application prefers. If true, the function writes a \thunk
that refers to its ``consequent'' input. Otherwise, the function
writes a \thunk that refers to its ``alternative'' input.

\textbf{Example VI: short-circuiting ``and''.} To implement a
short-circuiting \texttt{and} operator, the user writes a function
that expects one strict input and one lazy input, and writes
one output. The function returns a \thunk that expresses \texttt{if(\#0,
  if(\#1, true, false), false)}.

\textbf{Example VII: currying.} The \texttt{curry} function transforms
a function $f$ of two arguments into a function $h$ of one argument,
whose value is another function $g$ of one argument such that $h(x)(y) =
g(y) = f(x,y)$.

To implement this, the user creates two Wasm functions: \texttt{curry} and \texttt{partial-apply}.

The \texttt{curry} function expects one strict input, corresponding to
$f$.  The output is an \encode representing $h$. The function is
\texttt{partial-apply} and there is one strict input, naming $f$.

The \texttt{partial-apply} function expects two strict inputs,
representing $f$ and $x$. The output is an \encode representing
$g$. The function is $f$, and there is one strict input, naming $x$.

Because \encodes can be used as functions and the inputs are appended,
this allows the result of \texttt{curry($f$)} to be used as a function
applied to $x$, which in turn can be applied to $y$. The inputs could
also be provided in a different order, but then the $g$ function would
need to return a thunk that reorders the inputs before giving them to
the original $f$.

\vspace{0.5\baselineskip}
{\Large \textbf{5. Open questions}}

\begin{enumerate}[topsep=0pt]
\item How to handle I/O (random numbers, time, filesystem and network access, user interaction)

\item When/how to garbage-collect from the storage or memoization cache

\item Execution policy language separate from definition of a
  computation (to describe data placement / parallelization policy,
  garbage-collection policy, and when to give read-write access to
  inputs to a function because a \valuex will not be referenced again)

\item Does the approach to lazy inputs really ensure that a \thunk can
  always be replaced by (1) any other \thunk that refers to the same
  \blob, or (2) the referred-to \blob, without ever altering the
  result of a computation? (We don't want a function to be able to
  know if the runtime has or hasn't applied some optimization.) Does
  the system also successfully prevent cyclic references (e.g., two \thunks that resolve to
  each other)?
  
\item \ldots
  
\end{enumerate}

\end{document}
